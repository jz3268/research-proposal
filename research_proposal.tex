\documentclass[sigconf,nonacm,screen,letterpaper,9pt]{acmart}
%\usepackage{amssymb,amsmath}
\pagenumbering{gobble}
\usepackage{ifxetex,ifluatex}
\ifnum 0\ifxetex 1\fi\ifluatex 1\fi=0 % if pdftex
  \usepackage[T1]{fontenc}
  \usepackage[utf8]{inputenc}
\else % if luatex or xelatex
  \ifxetex
    \usepackage{mathspec}
  \else
    \usepackage{fontspec}
  \fi
  \defaultfontfeatures{Ligatures=TeX,Scale=MatchLowercase}
  \newcommand{\euro}{€}
\fi
% use upquote if available, for straight quotes in verbatim environments
\IfFileExists{upquote.sty}{\usepackage{upquote}}{}
% use microtype if available
\IfFileExists{microtype.sty}{%
\usepackage{microtype}
\UseMicrotypeSet[protrusion]{basicmath} % disable protrusion for tt fonts
}{}
\usepackage{hyperref}
\PassOptionsToPackage{usenames,dvipsnames}{color} % color is loaded by hyperref
\hypersetup{unicode=true,
            pdfsubject={Controllable, Reliable, and Safe Ingress Routing
to the Cloud},
            pdfborder={0 0 0},
            breaklinks=true}
\urlstyle{same}  % don't use monospace font for urls
\usepackage{natbib}
\bibliographystyle{ACM-Reference-Format.bst}
\usepackage{graphicx,grffile}
\makeatletter
\def\maxwidth{\ifdim\Gin@nat@width>\linewidth\linewidth\else\Gin@nat@width\fi}
\def\maxheight{\ifdim\Gin@nat@height>\textheight\textheight\else\Gin@nat@height\fi}
\makeatother
% Scale images if necessary, so that they will not overflow the page
% margins by default, and it is still possible to overwrite the defaults
% using explicit options in \includegraphics[width, height, ...]{}
\setkeys{Gin}{width=\maxwidth,height=\maxheight,keepaspectratio}
\setlength{\emergencystretch}{3em}  % prevent overfull lines
\providecommand{\tightlist}{%
  \setlength{\itemsep}{0pt}\setlength{\parskip}{0pt}}
% % \setcounter{secnumdepth}{5}
% 
\usepackage{xspace} % auto-eats spaces as needed
\usepackage{verbatim} % to support to-dos / comment environment
\usepackage{url} % links to bib items
%\usepackage{amsmath, amssymb} % math formulas, etc.
\usepackage{graphicx} % PDF, EPS, and many other graphics formats
\usepackage{multirow} % easier tables
% \usepackage{subfig}
\usepackage{float}
\usepackage[nolist]{acronym}
% \usepackage[nointegrals]{wasysym}
\usepackage[svgnames,x11names]{xcolor}
\usepackage{tabularx}
\usepackage{xparse}
\usepackage{pifont}
\usepackage[T1]{fontenc}
% \usepackage{lmodern}
\usepackage{amsmath}
\usepackage{colortbl}
\usepackage{enumitem}
\usepackage[htt]{hyphenat} % fixes line breaks with \texttt
\usepackage{marvosym} % cross symbols
\usepackage[flushleft]{threeparttable}
\usepackage{caption}
\usepackage{subcaption}
\usepackage{dirtytalk}
% These algorithm packages aren't working for me
% algorithm block renders in (escaped) plain text
%\usepackage{algorithm}
%\usepackage{algorithmic}
\usepackage{algorithm,algpseudocode}

\PassOptionsToPackage{usenames,dvipsnames}{xcolor}
\usepackage{balance}
\usepackage{soul}
\definecolor{lightlightgray}{rgb}{.9, .9, .9}

\usepackage{cleveref}
\crefformat{section}{#2\S#1#3}
\Crefformat{section}{#2Section~#1#3}
\crefname{section}{\S}{\S\S}
\Crefname{section}{Section}{Sections}
\crefname{equation}{Eq.}{Eqs.}
\Crefname{equation}{Equation}{Equations}
\crefname{table}{Table}{Tables}
\Crefname{table}{Table}{Tables}
\crefname{figure}{Fig.}{Figs.}
\Crefname{figure}{Figure}{Figures}

%\usepackage{authblk}
%
%\makeatletter
%\renewcommand\AB@affilsepx{, \protect\Affilfont}
%\makeatother
%\renewcommand\Authands{, }

\def\UrlBreaks{\do\/\do-\do.}
\def\UrlNoBreaks{\do:}

% subheaders
\newtheorem{Theorem}{Theorem}
\newtheorem{Lemma}{Lemma}
\newtheorem{Problem}{Problem}
\newtheorem{Definition}{Definition}

% caption setup
\usepackage{caption}
\captionsetup[table]{position=bottom, skip=4pt}
\captionsetup[figure]{position=bottom, skip=1pt}
%\setlength{\textfloatsep}{0.7\baselineskip plus 0.2\baselineskip minus 0.5\baselineskip}

% \usepackage{caption}
\setlength{\abovecaptionskip}{1pt}
\setlength{\belowcaptionskip}{-6pt}
% \captionsetup[table]{position=bottom, skip=4pt}
\renewcommand{\captionfont}{\small}

% float parameters
% needed to create an algorithm float
% \usepackage{float}
% \renewcommand{\topfraction}{0.99}
% \renewcommand{\dbltopfraction}{0.99}
% \renewcommand{\bottomfraction}{0.99}
% \renewcommand{\floatpagefraction}{0.99}
% \renewcommand{\dblfloatpagefraction}{0.99}
% \setcounter{totalnumber}{99}
% \setcounter{topnumber}{99}
% \setcounter{dbltopnumber}{99}

% \usepackage{subfig}

% to make tabulars easier
% from
% https://tex.stackexchange.com/questions/2441/how-to-add-a-forced-line-break-inside-a-table-cell
% v is one of t,c,b and h one of l,c,r
% \thead[bc]{blah \\ blah}
\usepackage{makecell}
\renewcommand\theadfont{\bfseries}

\usepackage{siunitx}
\sisetup{
  tight-spacing = true
}

\usepackage{pdflscape}

%% conference information
%\copyrightyear{2021}
%\acmYear{2021}
%\setcopyright{acmcopyright}\acmConference[SIGCOMM '21]{ACM SIGCOMM 2021 Conference}{August 23--27, 2021}{Virtual Event, USA}
%\acmBooktitle{ACM SIGCOMM 2021 Conference (SIGCOMM '21), August 23--27, 2021, Virtual Event, USA}
%\acmPrice{15.00}
%\acmDOI{10.1145/3452296.3472891}
%\acmISBN{978-1-4503-8383-7/21/08}

\hyphenation{name-spaces}

%% anonymization
\newif\ifisanon
\isanontrue

% \newcommand{\unicast}[1]{{\color{green}{#1}}}
\newcommand{\unicast}[1]{{{#1}}}

% \newcommand{\change}[2]{{\color{blue}{#2 (#1)}}} %% For the reviewer's convenience
\newcommand{\change}[2]{{#2}}


\newcommand\site{site\xspace}
\newcommand\Site{Site\xspace}
\newcommand\sites{sites\xspace}
\newcommand\Sites{Sites\xspace}

% \newcommand\peering{PEERING}

\newcommand\RA{\texttt{reactive\--anycast}\xspace}
\newcommand\PP{\texttt{proactive\--prepending}\xspace}
\newcommand\PS{\texttt{proactive\--superprefix}\xspace}

\newcommand\SRA{\texttt{safer\--reactive\--anycast}\xspace}
\newcommand\SPP{\texttt{controlled\--proactive\--prepending}\xspace}

\newcommand\SRAone{\texttt{selective\--reactive\--anycast (1 site)}\xspace}
\newcommand\SRAcap{\texttt{selective\--reactive\--anycast (capacity)}\xspace}
\newcommand\SPPone{\texttt{selective\--proactive\--prepending (1 site)}\xspace}
\newcommand\SPPall{\texttt{selective\--proactive\--prepending (all sites)}\xspace}


\newcommand\AnycastSubprefix{\texttt{reactive\--subprefix\--announcement}\xspace}
\newcommand\PrependSite{\texttt{reactive-prepended-announcement}\xspace}

%% Colors for impossible deal table
\newcommand\red{\color[HTML]{CC0000}}
\newcommand\green{\color[HTML]{38761D}}
\newcommand\yellow{\color[HTML]{F1C232}}
\newcommand\blue{\color[HTML]{3393FF}}


%% Getting rid of periods in bib entries
\newcommand{\killpunct}[1]{}

\newcommand\eat[1]{}

%% System name
\newcommand\thepeering{\textsc{Peering}\xspace}
\newcommand\peering{\textsc{Peering}\xspace}
%\newcommand\peering{\textsc{BGPPlatform}\xspace}
\newcommand\vbgp{\texttt{vBGP}\xspace}
\newcommand\testbed{platform\xspace}
\newcommand\testbeds{platforms\xspace}

%%% Macros for convenience
%% References
\newcommand{\secref}[1]{\S\ref{sec:#1}}
\newcommand{\figref}[1]{Figure~\ref{fig:#1}}
\newcommand{\tabref}[1]{Table~\ref{tab:#1}}
%% to refer to lines in graphs
\newcommand{\linename}[1]{\emph{#1}\xspace}
\newcommand{\parab}[1]{\smallskip\noindent {\bf #1}}

%% Common latin terms
\newcommand{\etc}{\emph{etc.}\xspace}
\newcommand{\ie}{\emph{i.e.,}\xspace}
\newcommand{\eg}{\emph{e.g.,}\xspace}
\newcommand{\etal}{\emph{et al.}\xspace}
\newcommand{\aka}{a.k.a\xspace}

%% editing notes
\newcommand\ekb[1]{{\color{blue}[ekb: #1]}}
\newcommand{\tbd}[1]{{\color{red}{\bf TBD: #1}}}

% \definecolor{shadecolor}{RGB}{150,150,150}
\newcommand{\lineboxblacktext}[2]{\colorbox{#1}{\texttt{#2}}}
\newcommand{\lineboxwhitetext}[2]{\colorbox{#1}{\color{white}{\texttt{#2}}}}

%\newcommand\new[1]{#1}
%\newcommand\cut[1]{}
\definecolor{orange}{rgb}{1.0, 0.31, 0.0}
\definecolor{rowgray}{gray}{.8}
%\newcommand\edit[2]{#2}
%\newcommand\new[1]{{#1}}
%\newcommand\cut[1]{{#1}}

% terms
\newcommand\noescape[1]{#1}

\newcommand{\rightdownarrow}{\mathrel{\scalebox{1}[-1]{$\Rsh$}}}
\newcommand{\cmark}{\color{green}\ding{51}}
\newcommand{\xmark}{\color{red}\ding{55}}
\newcommand{\metroas}{$\left<\texttt{region, AS}\right>$\xspace}
\newcommand{\metroasroot}{$\left<\texttt{region, AS, root}\right>$\xspace}
\newcommand{\fe}{front-end\xspace}
\newcommand{\feplural}{front-ends\xspace}
\newcommand{\capfe}{Front-end\xspace}
\newcommand{\capfeplural}{Front-ends\xspace}

\NewDocumentCommand{\rotseventy}{O{70} O{1em} m}{\makebox[#2][l]{\rotatebox{#1}{#3}}}


% De-Anonymized Commands
\newcommand\ISI{\textsc{the Information Sciences Institute (ISI) at USC}\xspace}
\newcommand\ISIshort{\textsc{ISI}\xspace}
\newcommand\thecdn{CDN\xspace}
\iffalse
% Anonymized Commands
\newcommand\ISIone{a research lab in a university in the United States\xspace} % introductory reference
\newcommand\ISItwo{the research lab\xspace} % more general references
\newcommand\ISIthree{a research lab\xspace}
\newcommand\thecdn{\texttt{ACDN}\xspace}
\fi


\title{Research Proposal}
\subtitle{Controllable, Reliable, and Safe Ingress Routing to the Cloud}

\author{Jiangchen Zhu}
\affiliation{\institution{Columbia University}}

\date{}
\pagestyle{plain}

%\begin{CCSXML}
    <ccs2012>
        <concept>
            <concept_id>10003033.10003079.10011672</concept_id>
            <concept_desc>Networks~Network performance analysis</concept_desc>
            <concept_significance>500</concept_significance>
        </concept>
    </ccs2012>
\end{CCSXML}

%\ccsdesc[500]{Networks~Network measurement}

%\keywords{Anycast, unicast, routing, performance, availability, CDN.}



\begin{document}
% \acmYear{2022}\copyrightyear{2022}
% \acmConference[IMC '22]{ACM Internet Measurement Conference}{October 25--27, 2022}{Nice, France}
% \acmBooktitle{ACM Internet Measurement Conference (IMC '22), October 25--27, 2022, Nice, France}
%\acmPrice{15.00}
%\acmDOI{10.1145/3517745.3561421}
%\acmISBN{978-1-4503-9259-4/22/10}

\maketitle


\setlength{\footskip}{40pt}
\pagenumbering{arabic}

\hypertarget{introduction}{%
\section{Introduction}\label{introduction}}

Clouds run numerous applications demanding low latency and high
availability and serve clients from many geographically distributed
\textit{sites}. To meet diverse objectives under fluctuating network
conditions - such as reducing latency, load balancing between sites and
routes, and ensuring fast failover during failures - the cloud needs
complete and timely control over the routes its clients use to reach its
sites, i.e., \textit{ingress routing} \tbd{cite}.

Two protocols are crucial for clients to reach the cloud: DNS and BGP.
The clients first learn an IP address to access the cloud service (a DNS
record) from the DNS server. When accessing the cloud, the packets
destined to that address are sent along routes decided by BGP on the
Internet. However, Both DNS and BGP have a number of known problems that
make ingress routing control challenging.

DNS records are cached by clients' recursive resolvers, applications,
and operating systems, which delays DNS updates on the client side. This
hurts the cloud's availability when a previously cached IP address
becomes unreachable due to failures. Each DNS record carries a
time-to-live (TTL) value, determining how long a DNS record should be
cached before expiring. Setting a low TTL causes clients to query the
DNS server more frequently, which can slow down applications. Moreover,
a shorter TTL value does not ensure timely DNS updates because many
applications disrespect the TTL and continue using expired DNS records.
Our analysis indicates that 13\% of client connections are started after
the DNS caches have expired, at median beginning 56 seconds after
expiration. Specifically for cloud traffic, between 20-85\% of traffic
occurs more than a minute after the DNS TTL has expired.

BGP decides the path a client takes to access the cloud, but this
decision is jointly made by the client network and others on the
Internet, each following their own routing policies, thus being out of
the cloud's control. BGP was not designed with the cloud's objectives,
such as minimizing latency and load balancing, in mind, thus the absence
of cloud's control over ingress routes leads to several issues. For
instance, BGP may select routes with suboptimal performance, resulting
in path inflation
\cite{painter, anycast_matt, anycast_ppp, imc-jc-deanony, tango}.
Moreover, load balancing becomes challenging when excessive clients
converge on identical routes or target the same site
\cite{tipsy, flavel2015fastroute}. BGP also suffers from convergence
issues: BGP updates cause networks to reselect routes, potentially
taking minutes to finalize their decisions, which results in higher
packet loss and latency
\cite{bgp-conv, a-measurement-study-on-the-impact}. Furthermore, rapidly
updating BGP routing on a global scale contradicts operational best
practices, as highlighted in a recent Google study \cite{capa}. Such
operations are deemed \emph{unsafe}, as any misconfiguration can quickly
spread worldwide, triggering cascading failures. This significantly
constrains the cloud's ability to safely respond to site failures.

Though these limitations are longstanding, clouds still lack universally
applicable and deployable solutions for controlling ingress routing.
Systems like Google's Espresso and Facebook's EdgeFabric have been
developed to optimize egress routes (from cloud to clients),
demonstrating the significance of route selection control for cloud
\cite{yap2017taking, schlinker2017engineering}. Controlling egress
routes is relatively straightforward since the cloud itself makes the
route decisions. In contrast, ingress route control remains challenging
because it depends on client networks, which are outside the cloud's
control. For ingress routing, two BGP announcement strategies, unicast
(with DNS-based redirection) and anycast, are commonly utilized.
However, these methods suffer from the limitations in DNS and BGP
protocols. Specifically, unicast is hindered by DNS caching, which
delays the failover of clients when a site fails. Anycast, on the other
hand, compromises the cloud's control over which site clients are
directed to, resulting in suboptimal performance and inadequate load
balancing.

Some recent work improves ingress routing control but requires
collaboration from customer networks or application developers. Systems
such as PAINTER and TANGO can only be deployed at collaborative customer
networks \cite{painter,tango-nsdi}. Solutions such as application-based
redirect and multi-path transport protocols require application support,
and their initial connection still uses DNS and BGP so their limitations
still exist.

Moreover, studying cloud routing problems poses significant challenges
for academic researchers, primarily because they lack the ability to
test their routing solutions in real cloud networks on the real
Internet. A typical cloud infrastructure spans tens to hundreds of sites
worldwide, each connecting to hundreds of peers. While some testbeds
enable researchers to conduct BGP routing experiments, their scope and
scale fall short of faithfully emulating a cloud network
\cite{tangled, schlinker19peering}. Conversely, simulating the Internet
within a laboratory setting often fails to yield compelling results due
to the complexities of accurately replicating both an accurate Internet
topology and the networks' intricate routing policies, which are
critical yet often undisclosed components of Internet routing
\tbd{cite}.

My contributions have played a crucial role in advancing academic
research on Internet routing at cloud scale and in the development of
practical techniques to improve cloud ingress routing control, which
were previously unattainable for cloud networks. These techniques adhere
to existing Internet protocols and do not require external collaboration
for easy deployment. Instead, they smartly leverage underutilized
variables within these protocols that have rarely been considered in the
context of cloud ingress routing.

\begin{itemize}[leftmargin=*]
    %\setlength{\itemindent}{-2\em}
\item


 \textbf{Expanding PEERING testbed to cloud scale} (\cref{sec:peering_vultr}). I collaborated with Vultr to expand PEERING \cite{schlinker19peering}, a BGP routing testbed, to cloud scale, enabling \textit{selective} BGP routing updates from 32 global locations to more than 5000 peers with customizable attributes such as AS path and BGP communities. This expansion makes realistic cloud routing research possible. For instance, a recent SIGCOMM paper leveraged this expanded testbed to develop and assess new ingress routing solutions for clouds \cite{painter}. Another study, recently published in NSDI, adopted a similar methodology, though researchers had to coordinate directly with the cloud provider to configure BGP \cite{tango-nsdi}. The expanded PEERING footprint is set to greatly simplify future research efforts in this field.


\item 
\textbf{Establishing fundamental tradeoffs in cloud ingress routing and developing techniques that achieve previously unattainable tradeoffs} (\cref{sec:pareto}). While the currently employed unicast and anycast techniques fall short of meeting certain objectives due to the limitations of DNS and BGP, it had been unclear whether a fundamental tradeoff is inherent in cloud ingress routing or if an "ideal" technique could exist. I proved an unavoidable tradeoff among control, availability, and operational safety in designing ingress routing solutions, a decision that must be tailored to a cloud's specific business needs. I then developed and evaluated new techniques that combine the strengths of existing methods, pushing them closer to the (unattainable) ideal.


\item 
\textbf{Achieving cloud routing objectives with higher ingress route control} (\cref{sec:community}).
I propose to develop new systems that take cloud's ingress routing objectives and network conditions as input to optimize the prefix announcement strategies from its sites. The objectives include minimizing the overall clients latency and load balancing between provider links. This requires the cloud to have rich ingress route control by tailoring where and to whom these announcements are made. Controlling what BGP route a directly neighboring network receives is straightforward, but influencing networks beyond a one-hop distance, where they are free to select from various BGP announcements they have learnt, is challenging. Fortunately, BGP communities allow the cloud to direct how neighboring networks further propagate its announcements, providing a mechanism by which the cloud can potentially further influence the routes of distant clients to increase its ingress control. \eat{ For example, the cloud can direct a neighboring network to suppress announcements to certain networks to encourage alternative routing paths.} However, the complexity of BGP communities, with their arbitrary formats and different types of actions documented on network-specific websites, makes manual interpretation and verification time-consuming and uncertain.  Automating the learning and verification of BGP communities is a critical first step towards providing more controlled ingress routing options to clients. Future challenges include assessing the benefit of a diverse set of ingress routes in achieving routing objectives and creating a scalable framework that integrates various BGP communities to optimize routing for many client networks. \eat{To effectively manage clients' ingress routes using BGP communities, a system first needs to predict the impact of specific BGP community actions, which vary widely (e.g., affecting one specific network or a set of networks, affecting global or regional networks). It must then correlate these predicted routes with specific objectives, such as minimizing client latency or balancing network load. The above processes require extensive measurements, including latency measurement and client preferences for various ingress routes. The final step involves searching among a vast set of potential ingress routing options to optimize the announcement strategy for each IP prefix to achieve the desired objectives. A practical method might involve a greedy algorithm, selecting the most beneficial BGP configuration update iteratively and stopping when additional changes yield marginal benefits.}




 
\end{itemize}

\hypertarget{related-work}{%
\section{Related Work}\label{related-work}}

\emph{Limitations of DNS.} One prior work shows how using DNS can
control clients to specific sites, achieving ingress site control
\cite{end-user-mapping}. Much has discussed the challenges of DNS TTL
violation in cloud routing availability
\cite{cache-me-if-you-can, dns-performance-caching, dns-in-context}.

\emph{Limitations of BGP.}

\hypertarget{details-of-my-contributions}{%
\section{Details of My
Contributions}\label{details-of-my-contributions}}

I aim to develop techniques that enable the cloud to meet its ingress
routing objectives, such as improved latency and availability. Achieving
these goals requires the cloud to have timely and flexible control over
how external networks select their routes, a level of control that was
previously unattainable without collaboration with them. To aid the
research community in studying these routing challenges, I first
upgraded a testbed to match the scale of a medium-sized cloud
(\cref{sec:peering_vultr}). My proposed techniques continue to utilize
existing Internet protocols, which allows for straightforward
deployment. However, these techniques uniquely leverage underexplored
variables within these protocols, enhancing their effectiveness
(\cref{sec:pareto}, \cref{sec:community}).

\hypertarget{expanding-the-peering-testbed}{%
\subsection{Expanding the PEERING
Testbed}\label{expanding-the-peering-testbed}}

\label{sec:peering_vultr} The PEERING testbed allows researchers to make
customizable BGP announcements selectively to its neighboring networks,
enhancing the study of BGP routing \cite{schlinker19peering}. Despite
its utility in numerous innovative studies
\cite{painter, imc-jc-deanony, vermeulen2022internet, anycast-agility, rtt-monitor},
PEERING's scale, 14 global locations and 5 IXP connections, is limited
for routing studies with larger scale (e.g., cloud). Vultr, a cloud
provider, enhances this by offering a ``bring your own IP'' service.
This service lets its customers announce their IP spaces and choose
which neighboring networks receive these announcements by tagging the
relevant BGP communities. It also allows modification of attributes such
as AS path and BGP communities in the BGP updates.

After integrating PEERING infrastructure with Vultr servers, researchers
can now make BGP announcements from 32 additional locations, reaching
over 5000 neighboring networks selectively. The expanded locations span
several continents: 11 in North America, 2 in South America, 8 in Asia,
8 in Europe, 2 in Oceania, and 1 in Africa.

This expanded footprint provides a unique platform for addressing cloud
routing challenges previously unexplored. A recent SIGCOMM paper
utilized this enhanced testbed for cloud-enterprise collaborative
routing studies \cite{painter}. The expanded capabilities can also
retroactively enhance many studies that previously utilized the PEERING
testbed \cite{imc-jc-deanony, anycast-agility}.

\hypertarget{establishing-fundamental-tradeoffs-in-cloud-ingress-routing-and-developing-better-techniques}{%
\subsection{Establishing Fundamental Tradeoffs in Cloud Ingress Routing
and Developing Better
Techniques}\label{establishing-fundamental-tradeoffs-in-cloud-ingress-routing-and-developing-better-techniques}}

\label{sec:pareto} The clouds have employed two strategies when routing
their clients to their sites: unicast with DNS-based redirection and
anycast. In unicast, each site announces a unique IP prefix and the DNS
server returns clients an address within the prefix of a specific site
to redirect the clients to that site. Despite the complete cloud control
on clients redirection, the continued use of DNS caching after
expiration makes quick DNS updates on client side impossible, delaying
the failover when a site fails for minutes and losing availability. On
the other hand, in anycast, each site announces the same IP prefix, but
the cloud compromises its \emph{control} on which site a client will be
routed to since the decision is made by the client networks and other
networks instead of the cloud itself. Consequently, clients can end up
being routed to geographically distant sites, experiencing increased
latency. The lack of control also complicates load balancing between
sites.

I define three key \emph{goals} for cloud ingress routing:
\emph{control}, \emph{availability} and \emph{safety}. Control refers to
the cloud's capability of routing clients to any site it wants.
Availability refers to quickly rerouting clients from a failed site.
Safety refers to avoiding updating BGP configurations on healthy sites
to minimize the risk of cascading failure as quick global routing
reconfiguration contradicts operational best practices. I first observe
a fundamental tension: while control benefits from each site announcing
a unique IP prefix, guaranteeing any packets destined to the prefix to
arrive at the site, availability benefits from multiple sites announcing
the same prefix, rerouting clients to healthy sites without waiting for
the slow DNS update during site outage. Maximizing both control and
availability requires changing BGP announcements upon site failure,
compromising safety. I present a more formal proof in a recent
submission.

Following this fundamental tradeoff, I develop and evaluate four
techniques using the expanded PEERING testbed on the real Internet.
\Cref{tab:summary} summarizes all comparisons among my proposed
techniques with respect to CDN routing goals. The new techniques achieve
previously unattainable combinations of goals, with no existing
techniques that can beat them on one goal without compromising on
another. Together, they form a new set of ``best techniques'' for CDN
routing. The preliminary work on these techniques (with worse tradeoffs)
was published in IMC and awarded a best short paper. The refined
techniques are presented in a recently submitted paper.

\begin{table}[]
\small
\label{tab:summary}
\begin{tabular}{l|lll}
\multicolumn{1}{c|}{Technique}                 & Control                      & Availability                  & Safety                        \\ \hline
anycast                                        & {\color[HTML]{FE0000} low}   & {\color[HTML]{036400} high}   & {\color[HTML]{036400} high}   \\ \hline
unicast                                        & {\color[HTML]{036400} high}  & {\color[HTML]{FE0000} low}    & {\color[HTML]{036400} high}   \\ \hline
\eat{capacity-aware-proactive-superprefix} My technique A          & {\color[HTML]{036400} high}  & {\color[HTML]{FFCB2F} medium} & {\color[HTML]{036400} high}   \\ \hline
\eat{capacity-aware-safer-reactive-anycast} My technique B        & {\color[HTML]{036400} high}  & {\color[HTML]{036400} high}   & {\color[HTML]{FFCB2F} medium} \\ \hline
\eat{capacity-aware-controlled-proactive-prepending} My technique C & {\color[HTML]{036400} high-} & {\color[HTML]{036400} high-}  & {\color[HTML]{036400} high}   \\ \hline
\eat{capacity-aware-controlled-proactive-depref}  My technique D   & {\color[HTML]{036400} high-} & {\color[HTML]{036400} high}   & {\color[HTML]{036400} high}  
\end{tabular}
\caption{Ingress routing technique tradeoffs among three goals: control, availability and safety. Minus signs are used to indicate being slightly worse in achieving a certain goal. }
\label{tab:summary}




\end{table}

\hypertarget{achieving-cloud-routing-objectives-with-higher-ingress-route-control}{%
\subsection{Achieving Cloud Routing Objectives with Higher Ingress Route
Control}\label{achieving-cloud-routing-objectives-with-higher-ingress-route-control}}

\label{sec:community}

While the techniques in \Cref{sec:pareto} manage to achieve the control
on which \emph{site} clients ingress the cloud without compromising much
availability or safety, they have not yet achieved \emph{ingress route}
control. I plan to develop systems that take the cloud's routing
objective as the input and decide how BGP routes are announced from each
site. The objectives can vary from minimizing client latency overall or
based on traffic priority, load balancing between links/paths, and
facilitating failover after site/link outage. To achieve these
objectives, cloud first needs rich control on the ingress route the
clients take. Although it is straightforward to control the ingress
routing for cloud's neighboring networks by making tailored
announcements directly to them, it is hard to achieve this for many more
networks that are beyond a one-hop distance since they are free to
select any BGP routes learnt from other networks.

Fortunately, the BGP community provides a powerful yet previously
unexplored mechanism for influencing the routes of distant client
networks. The cloud can direct how its directly neighboring networks
further propagate its announcements to distant client networks by
tagging the routes with tailored BGP communities. However, BGP
communities are 32-bit values with arbitrary formats and
network-specific interpretations documented on their websites. For
example, two provider networks AS3257 and AS2914 have BGP communities
with the same interpretation ``do not announce to peers'' but with
drastically different values 65535:65284 and 2914:429. The types of
actions encoded in BGP communities also vary significantly between
networks. For example, 65501:nnn is a BGP community of AS2914 which
means ``prepend to a specific peer nnn 1x'', allowing its customers to
specify any peer network to apply the action. In contrast, the BGP
communities from AS3491 do not support applying to any specific network,
but instead have one particular value for each of its peer networks
(e.g., 3491:60041 means ``prepend to AS174 1x'' and there are numerous
other values for other networks).

Due to the diversity and a large volume of BGP communities, manual
collection and interpretation of BGP communities is time-consuming and
unscalable. Automating the learning and verification of BGP communities
is the first challenge in realizing a system to achieve ingress route
control. I have developed a set of techniques that leverage the latest
developments in NLP tools to interpret the semantics of BGP communities
and designed tailored Internet measurements to verify their
interpretation. Here are the remaining challenges and planned
milestones:

To manage clients' ingress routes effectively using BGP communities, the
system must first be capable of predicting the varied impacts of
specific BGP community actions on clients' ingress routes. These can
range widely - for example, some may target a single network while
others affect multiple; some have global impacts, while others are
regional. It then needs to align these predicted impacts with the
objectives, such as minimizing client latency or load balancing between
links and sites. The above processes demand extensive measurements to
measure or predict route latencies as well as client networks'
preferences for different exposable ingress routes. The next step
involves searching among a vast set of potential ingress routing options
to optimize the announcement strategy for each IP prefix to achieve the
desired objectives. A practical method might involve a greedy algorithm,
selecting the most beneficial BGP configuration update iteratively and
stopping when additional changes yield marginal benefits.

\hypertarget{conclusion}{%
\section{Conclusion}\label{conclusion}}

Existing routing techniques employed by the cloud are insufficient in
achieving their routing objectives such as latency, load balancing and
reliability, due to the limitations of DNS and BGP. The expansion of the
PEERING testbed to cloud scale benefits my study of cloud ingress
routing as well as the research community. I then propose and evaluate
new techniques that significantly improve cloud's control over clients'
ingress routes. My techniques for improving cloud ingress routing
achieve previously unattainable tradeoffs and a fine-grained control.

\iffalse

END OF PAPER

% Do don't know why, but the pandocs pipeline started to include
% google docs comments at the end of the body.  So to fix that
% problem, the google doc requires an \iffalse as the last statement,
% then the following \fi will stop that comment block, effectively
% removing the comments from the final PDF.
\fi
{\small\balance\bibliography{mybib.bib}}

\end{document}
