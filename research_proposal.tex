\documentclass[sigconf,nonacm,screen,letterpaper,10pt]{acmart}
%\usepackage{amssymb,amsmath}
\pagenumbering{gobble}
\usepackage{ifxetex,ifluatex}
\ifnum 0\ifxetex 1\fi\ifluatex 1\fi=0 % if pdftex
  \usepackage[T1]{fontenc}
  \usepackage[utf8]{inputenc}
\else % if luatex or xelatex
  \ifxetex
    \usepackage{mathspec}
  \else
    \usepackage{fontspec}
  \fi
  \defaultfontfeatures{Ligatures=TeX,Scale=MatchLowercase}
  \newcommand{\euro}{€}
\fi
% use upquote if available, for straight quotes in verbatim environments
\IfFileExists{upquote.sty}{\usepackage{upquote}}{}
% use microtype if available
\IfFileExists{microtype.sty}{%
\usepackage{microtype}
\UseMicrotypeSet[protrusion]{basicmath} % disable protrusion for tt fonts
}{}
\usepackage{hyperref}
\PassOptionsToPackage{usenames,dvipsnames}{color} % color is loaded by hyperref
\hypersetup{unicode=true,
            pdfsubject={Controllable, Reliable, and Safe Ingress Routing
to the Cloud},
            pdfborder={0 0 0},
            breaklinks=true}
\urlstyle{same}  % don't use monospace font for urls
\usepackage{natbib}
\bibliographystyle{ACM-Reference-Format.bst}
\usepackage{graphicx,grffile}
\makeatletter
\def\maxwidth{\ifdim\Gin@nat@width>\linewidth\linewidth\else\Gin@nat@width\fi}
\def\maxheight{\ifdim\Gin@nat@height>\textheight\textheight\else\Gin@nat@height\fi}
\makeatother
% Scale images if necessary, so that they will not overflow the page
% margins by default, and it is still possible to overwrite the defaults
% using explicit options in \includegraphics[width, height, ...]{}
\setkeys{Gin}{width=\maxwidth,height=\maxheight,keepaspectratio}
\setlength{\emergencystretch}{3em}  % prevent overfull lines
\providecommand{\tightlist}{%
  \setlength{\itemsep}{0pt}\setlength{\parskip}{0pt}}
% % \setcounter{secnumdepth}{5}
% 
\input{packages}
\input{macros}

\title{Research Proposal}
\subtitle{Controllable, Reliable, and Safe Ingress Routing to the Cloud}

\author{Jiangchen Zhu}
\affiliation{\institution{Columbia University}}

\date{}
\pagestyle{plain}

%\include{ccs_keywords}


\begin{document}
% \acmYear{2022}\copyrightyear{2022}
% \acmConference[IMC '22]{ACM Internet Measurement Conference}{October 25--27, 2022}{Nice, France}
% \acmBooktitle{ACM Internet Measurement Conference (IMC '22), October 25--27, 2022, Nice, France}
%\acmPrice{15.00}
%\acmDOI{10.1145/3517745.3561421}
%\acmISBN{978-1-4503-9259-4/22/10}

\maketitle


\setlength{\footskip}{40pt}
\pagenumbering{arabic}

\hypertarget{introduction}{%
\section{Introduction}\label{introduction}}

Cloud providers run numerous applications demanding low latency and high
availability and serve clients from many geographically distributed
\textit{sites}. To meet diverse objectives under fluctuating network
conditions - such as reducing latency, load balancing between sites and
routes, and ensuring fast failover during failures - the Cloud needs
complete and timely control over the routes its clients use to reach its
sites, i.e., \textit{ingress routing} \tbd{cite}.

Two protocols are crucial for clients to reach the Cloud: DNS and BGP.
The clients first learn an IP address to access the Cloud service (a DNS
record) from the DNS server. When accessing the Cloud, the packets
destined to that address are sent along routes decided by BGP on the
Internet. However, Both DNS and BGP have a number of known problems that
make ingress routing control challenging.

DNS records are cached by clients' recursive resolvers, applications,
and operating systems, which delays DNS updates on the client side. This
hurts the Cloud's availability when a previously cached IP address
becomes unreachable due to failures. Each DNS record carries a
time-to-live (TTL) value, determining how long a DNS record should be
cached before expiring. Setting a low TTL causes clients to query the
DNS server more frequently, which can slow down applications. Moreover,
a shorter TTL value does not ensure timely DNS updates because many
applications disrespect the TTL and continue using expired DNS records.
Our analysis indicates that 13\% of client connections are started after
the DNS caches have expired, at median beginning 56 seconds after
expiration. Specifically for Cloud traffic, between 20-85\% of traffic
occurs more than a minute after the DNS TTL has expired.

BGP decides the path a client takes to access the Cloud, but this
decision is jointly made by the client network and others on the
Internet, each following their own routing policies, thus being out of
the Cloud's control. BGP was not designed with the Cloud's objectives,
such as minimizing latency and load balancing, in mind, thus the absence
of Cloud's control over ingress routes leads to several issues. For
instance, BGP may select routes with suboptimal performance, resulting
in path inflation
\cite{painter, anycast_matt, anycast_ppp, imc-jc-deanony, tango}.
Moreover, load balancing becomes challenging when excessive clients
converge on identical routes or target the same site
\cite{tipsy, flavel2015fastroute}. BGP also suffers from convergence
issues: BGP updates cause networks to reselect routes, potentially
taking minutes to finalize their decisions, which results in higher
packet loss and latency
\cite{bgp-conv, a-measurement-study-on-the-impact}. Furthermore, rapidly
updating BGP routing on a global scale contradicts operational best
practices, as highlighted in a recent Google study \cite{capa}. Such
operations are deemed \emph{unsafe}, as any misconfiguration can quickly
spread worldwide, triggering cascading failures. This significantly
constrains the Cloud's ability to safely respond to site failures.

Though these limitations are longstanding, Clouds still lack universally
applicable and deployable solutions for controlling ingress routing.
Systems like Google's Espresso and Facebook's EdgeFabric have been
developed to optimize egress routes (from Cloud to clients),
demonstrating the significance of route selection control for Cloud
\cite{yap2017taking, schlinker2017engineering}. Controlling egress
routes is relatively straightforward since the Cloud itself make the
route decisions. In contrast, ingress route control remains challenging
because it depends on client networks, which are outside the cloud's
control. For ingress routing, two BGP announcement strategies, unicast
(with DNS-based redirection) and anycast, are commonly utilized.
However, these methods suffer from the limitations in DNS and BGP
protocols. Specifically, unicast is hindered by DNS caching, which
delays the failover of clients when a site fails. Anycast, on the other
hand, compromises the Cloud's control over which site clients are
directed to, resulting in suboptimal performance and inadequate load
balancing.

Some recent work improves ingress routing control but requires
collaboration from customer networks or application developers. Systems
such as PAINTER and TANGO can only be deployed at collaborative customer
networks \cite{painter,tango-nsdi}. Solutions such as application-based
redirect and multi-path transport protocols require application support,
and their initial connection still uses DNS and BGP so their limitations
still exist.

Moreover, studying Cloud routing problems poses significant challenges
for academic researchers, primarily because they lack the ability to
test their routing solutions in real Cloud networks on the real
Internet. A typical Cloud infrastructure spans tens to hundreds of sites
worldwide, each connecting to hundreds of peers. While some testbeds
enable researchers to conduct BGP routing experiments, their scope and
scale fall short of faithfully emulating a Cloud network
\cite{tangled, schlinker19peering}. Conversely, simulating the Internet
within a laboratory setting often fails to yield compelling results due
to the complexities of accurately replicating both an accurate Internet
topology and the networks' intricate routing policies, which are
critical yet often undisclosed components of Internet routing.

My contributions have played a crucial role in advancing academic
research on Internet routing at Cloud scale and in the development of
practical techniques to improve Cloud ingress routing control, which
were previously unattainable for Cloud networks. These techniques adhere
to existing Internet protocols and do not require external collaboration
for easy deployment. Instead, they smartly leverage underutilized
variables within these protocols that have rarely been considered in the
context of Cloud ingress routing.

\begin{itemize}[leftmargin=*]
    %\setlength{\itemindent}{-2\em}
\item


 \textbf{Expanding PEERING Testbed to Cloud scale.} I collaborated with Vultr to expand PEERING \cite{schlinker19peering}, a BGP routing testbed, to Cloud scale, enabling \textit{selective} BGP routing updates from 30 global locations to about 5000 peers with customizable attributes such as AS path and BGP communities. This expansion makes realistic cloud routing research possible. For instance, a recent SIGCOMM paper leveraged this expanded testbed to develop and assess new ingress routing solutions for clouds \cite{painter}. Another study, recently published in NSDI, adopted a similar methodology, though researchers had to coordinate directly with the cloud provider to configure BGP \cite{tango-nsdi}. The expanded PEERING footprint is set to greatly simplify future research efforts in this field.


\item 
\textbf{Fundamental tradeoffs in Cloud ingress routing and a new Pareto Frontier.} While the currently employed unicast and anycast techniques fall short of meeting certain objectives due to the limitations of DNS and BGP, it had been unclear whether a fundamental tradeoff is inherent in Cloud ingress routing or if an "ideal" technique could exist. I demonstrated an unavoidable tradeoff among control, availability, and operational safety in designing ingress routing solutions, a decision that must be tailored to a Cloud's specific business needs. I then developed and evaluated new techniques that combine the strengths of existing methods, pushing them closer to the ideal.


\item 
\textbf{Improving ingress routing flexibility with BGP communities.} To control ingress routing, the Cloud decides the propagation of its BGP announcements across the Internet by tailoring where and to whom these announcements are made. Direct adjustments to neighboring networks are straightforward, but influencing networks beyond a one-hop distance, where they are free to select from various BGP announcements they learnt, is challenging. Fortunately, BGP communities allow the Cloud to direct how neighboring networks propagate its announcements, expanding ingress routing options for distant clients. \eat{ For example, a Cloud provider can direct a neighboring network to suppress announcements to certain networks to encourage alternative routing paths.} However, the complexity of BGP communities, with their arbitrary formats and meanings documented on network-specific websites, makes manual interpretation and verification time-consuming and uncertain. Automating the learning and verification of BGP communities is a critical first step towards providing more controlled ingress routing options to clients. Eventually, I aim to develop new systems that take Cloud's ingress routing objectives and network conditions as input to optimize announcement strategies from its sites.


\end{itemize}

\hypertarget{related-work}{%
\section{Related Work}\label{related-work}}

\hypertarget{details-of-my-contributions}{%
\section{Details of My
Contributions}\label{details-of-my-contributions}}

I aim to develop techniques that enable the Cloud to meet its ingress
routing objectives, such as improved latency and availability. Achieving
these goals requires the Cloud to have timely and flexible control over
how external networks select their routes, a level of control that was
previously unattainable without collaboration with them. To aid the
research community in studying these routing challenges, I first
upgraded a testbed to match the scale of a medium-sized Cloud
(\cref{sec:peering_vultr}). My proposed techniques continue to utilize
existing Internet protocols, which allows for straightforward
deployment. However, these techniques uniquely leverage underexplored
variables within these protocols, enhancing their effectiveness
(\cref{sec:pareto}, \cref{sec:community}).

\hypertarget{expanding-the-peering-testbed}{%
\subsection{Expanding the PEERING
Testbed}\label{expanding-the-peering-testbed}}

\label{sec:peering_vultr}

\hypertarget{fundamental-tradeoffs-in-cloud-ingress-routing-and-a-new-pareto-frontier}{%
\subsection{Fundamental Tradeoffs in Cloud Ingress Routing and a New
Pareto
Frontier}\label{fundamental-tradeoffs-in-cloud-ingress-routing-and-a-new-pareto-frontier}}

\label{sec:pareto}

\hypertarget{improving-ingress-routing-flexibility-with-bgp-communities}{%
\subsection{Improving Ingress Routing Flexibility with BGP
Communities}\label{improving-ingress-routing-flexibility-with-bgp-communities}}

\label{sec:community} a

\iffalse

END OF PAPER

% Do don't know why, but the pandocs pipeline started to include
% google docs comments at the end of the body.  So to fix that
% problem, the google doc requires an \iffalse as the last statement,
% then the following \fi will stop that comment block, effectively
% removing the comments from the final PDF.
\fi
{\small\balance\bibliography{mybib.bib}}

\end{document}
